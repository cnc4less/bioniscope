\chapter{Evaluation}

As detailed in~\cref{sec:initialCriteriaMet}, all initial criteria were met and
system testing verified this.

The aim of this project was to create a Digital Sampling Oscilloscope, and that
aim certainly succeeded. While it may not be the fastest or most accurate DSO
available it's relatively unique in that it communicates with an Android tablet.
The total cost (estimated to be about \pounds 35 if free samples were not
available) also comes to much less than other DSOs, even the cheapest of which
cost hundreds of pounds.

There were two other secondary aims of the project: to generate frequency
spectrums and to have basic logic analyser capability. As seen in the testing
and general screenshots, frequency spectrums are certainly generated. While
their accuracy was not numerically tested as in-depth as the waveforms, the
spectrums are generated from the waveforms using an industry-standard algorithm
so will have the same standard of accuracy.

Basic logic analyser functionality was also implemented. At the moment, only
timing diagrams can be displayed and no triggering functionality exists, which
makes it less useful than a standalone logic analyser but still much better than
a standard CRO. In the future, more advanced triggering could be implemented, as
well as things like packet decoding (so data packets can be decoded and
displayed all inside the logic analyser). This could all be implemented in
software, so the hardware would not need to be modified.

\section{Report Production}

This report was produced in \LaTeX. Whilst offering many advantages over
alternatives that could have been chosen (such as Microsoft Word), \LaTeX offers
superior typesetting, referencing and integration of mathematical equations and
figures.

For more information on \LaTeX, please consult a resource such as
\url{http://www.sharelatex.com}.

As indicated on the front cover, this report is licensed under a Creative
Commons Attribution-ShareAlike 4.0 International License. See
\url{https://creativecommons.org/licenses/by-sa/4.0/} for full details, but
essentially one is allowed to share and adapt this work, even commercially, as
long as attribution is given and the resultant work is distributed under the
same license. The circuit and PCB is also licensed under CC BY-SA 4.0, and the
Android application is licensed under the GNU General Public License 3.0 (the
`traditional' open source license).
